%----------------------------------------------------------------------------------------
%	SECTION TITLE
%----------------------------------------------------------------------------------------

\cvsection{Undergraduate Projects}

%----------------------------------------------------------------------------------------
%	SECTION CONTENT
%----------------------------------------------------------------------------------------

\begin{cventries}

%------------------------------------------------

\cventry
{\textbf{Tech Stack :} Unity, C\#, SpringBoot, React, MySQL} % Affiliation/role
{Hybrid Farm - Game Development} % Organization/group
{\href{https://github.com/Stealth-Devs-Org/Stealth-SDC-Sem6}{Github Link}} % Location
{Feb. 2024 - Jul. 2024} % Date(s)
{ % Description(s) of experience/contributions/knowledge
\begin{cvitems}
\item {This was a group project for the Software Design Competition module in semester 6.}
\item {The objective was to develop a game that motivates users to save energy by integrating with an energy demand management application. The game should encourage behaviors like reducing power consumption, utilizing renewable energy, and participating in demand response programs. The game's core lies in its ability to reflect changes based on player decisions.}
\item {We developed a 2.5D point-and-click strategic gameplay that enables the player to manage the production processes on a farm, make decisions to increase profit, and complete objectives in less time to secure a position on the leaderboard.}
\item {The game is integrated with an energy demand management API that provides real-time data on energy consumption and production. The game's environment changes based on the player's energy consumption and production, and the player will experience the impact of their decisions on the environment.}
\end{cvitems}
}

%------------------------------------------------

\cventry
{\textbf{Tech Stack :} Node-Red, JavaScript, HTML, Arduino} % Affiliation/role
{Vending Machines' Network IoT Project} % Organization/group
{\href{https://github.com/VijthanKurrshanth/iot-project-vending-machine-instance}{Github Link}} % Location
{Oct. 2023 - Dec. 2023} % Date(s)
{ % Description(s) of experience/contributions/knowledge
\begin{cvitems}
\item {This was a group project for the Internet of Things module in semester 5.}
\item {This project is a network of vending machines of a company that sells a defined set of different products.}
\item {The main idea is to connect the vending machines in a star topology to a main server (aka admin server),
to monitor the product availability in each vending machine and to provide customers with valuable information such as nearby locations, product availability in nearby locations, details of resupply schedules. Each vending machine is equipped with a convenient user interface for the customer where he/she can buy
products or inquire about info.}
\item {This implementation also benefits the company due to real-time monitoring and features like notification when products run low, and location-wise or product-wise stats, to efficiently schedule supply routines and to accurately identify the market value of products by analyzing the trends to make informed decisions on future marketing aspects.}
\end{cvitems}
}

%------------------------------------------------

\cventry
{\textbf{Tech Stack :} Arduino, Altium, SolidWorks} % Affiliation/role
{Colour Sensing Customizable Table Lamp} % Organization/group
{\href{https://github.com/VijthanKurrshanth/Color-Sensing-Lamp-Project/tree/main}{Github Link}} % Location
{Mar. 2023 - Jun. 2023} % Date(s)
{ % Description(s) of experience/contributions/knowledge
\begin{cvitems}
\item {This was an individual project for the Electronic Design Realization module in semester 4.}
\item {Designed and manufactured a table lamp that has an RGB LED lamp that is color customizable according to user inputs or according to sensed colors through the sensor. It has a small OLED display and three touch buttons for user input.}
\end{cvitems}
}

\cventry
{\textbf{Tech Stack :} C++} % Affiliation/role
{Flower Exchange Software} % Organization/group
{\href{https://github.com/VijthanKurrshanth/LSEG}{Github Link}} % Location
{Jun. 2023 - Sep. 2023} % Date(s)
{ % Description(s) of experience/contributions/knowledge
\begin{cvitems}
\item {This was a two-person group project for a 'C++ Workshop' offered by the 'London Stock Exchange Group (LSEG)'.}
\item {Designed a basic trading system, where traders can submit buy or sell orders for flowers via the Trader-Application and the Exchange-Application will process the incoming order against existing orders in the order container (known as Order Book) and do a full or partial execution.} 
\item {Every order is replied to with an Execution Report by the Exchange Application indicating the status of the order. Orders sometimes could be rejected due to quantity limitations, invalid flower type, etc.}
\end{cvitems}
}

%------------------------------------------------

\cventry
{\textbf{Tech Stack :} Cisco Packet Tracer} % Affiliation/role
{Design of Local Area Network and Routing Simulation} % Organization/group
{\href{https://github.com/VijthanKurrshanth/LAN-Design}{Github Link}} % Location
{Apr. 2023 - Jul. 2023} % Date(s)
{ % Description(s) of experience/contributions/knowledge
\begin{cvitems}
\item {This was a group project for the Communication Network Engineering module in semester 4.}
\item {Designed a backbone network for the University of Moratuwa and the internal network of one building (ENTC).} 
\item {Several design aspects such as backbone topology, IP addressing scheme for the network, bandwidth of each link, selection of active (switches/routers) and passive components, features/specifications of the routers/switches and bill of quantities for passive and active components were considered.}
\item {Developed the routing configuration for the backbone network using OSPF and the final design is simulated and tested using Cisco Packet Tracer software tool.}
\end{cvitems}
}

%------------------------------------------------

\cventry
{\textbf{Tech Stack :} GNURadio, GNUOctave} % Affiliation/role
{QPSK Digital Transceiver using GNURadio and BladeRF SDR} % Organization/group
{\href{https://github.com/VijthanKurrshanth/QPSK-Tranciever-GnuRadio}{Github Link}} % Location
{Nov. 2022 - Jan. 2023} % Date(s)
{ % Description(s) of experience/contributions/knowledge
\begin{cvitems}
\item {This was a group project for the Communication Design Project module in semester 3.}
\item {Implemented a point-to-point digital wireless communication system using GNURadio and bladeRF (software-defined radio).}
\item {Implemented forward error correction encoding, matrix interleaving for burst errors, packetizing with access codes, bit scrambling for ease of clock recovery, QPSK modulation and clock recovery technique using polyphase clock sync, linear equalizer and Costas loop.} 
\item {Successfully transmitted audio files, live voice, bitstream and image in simulation with channel models, and successfully transmitted audio files between two bladeRFs wirelessly.}
\end{cvitems}
}

%------------------------------------------------

\cventry
{\textbf{Tech Stack :} MultiSim, Altium, SolidWorks} % Affiliation/role
{Function Generator} % Organization/group
{\href{https://github.com/VijthanKurrshanth/Function_Generator}{Github Link}} % Location
{Nov. 2022 - Jan. 2023} % Date(s)
{ % Description(s) of experience/contributions/knowledge
\begin{cvitems}
\item {This was a group project for the Laboratory Practices module in semester 3.}
\item {The final product had to be designed to consist only of analog components.}
\item {Designed and manufactured a function generator that can generate sine, square, sawtooth, and triangular waves with an output amplitude of 0V to 10V, and an output frequency of 20 Hz to 20000 Hz. Square pulse waveform can output a variable pulse width (1\% to 99\%).}
\end{cvitems}
}

%------------------------------------------------

\end{cventries}